\documentclass{article}

\usepackage[utf8]{inputenc}
\usepackage[T1]{fontenc}
\usepackage[english]{babel}
\usepackage[document]{ragged2e}

\usepackage{amsfonts}
\usepackage{natbib}
\usepackage[dvipsnames]{xcolor}
\usepackage{graphicx}

\usepackage[fleqn]{amsmath}
\usepackage{amssymb}

\usepackage{minted}

\title{Logica difusa}
\author{Victor Gerardo Rodríguez Barragán}
\date{29 de Octubre de 2023}

\begin{document}
\maketitle
\begin{center}
    \includegraphics[width=0.8\textwidth]{/home/gerry/Documents/ceti/cetilogo.jpg}
\end{center}

\newpage
\justify
\textbf{Caso de aplicación de lógica difusa en el control de robots}
\vspace{0.5cm}\\
\textbf{Disciplina:} Ciencias de la computación
\vspace{0.5cm}\\
\textbf{Tipo de datos:} Datos sensoriales del robot, como la posición, la velocidad y la orientación.
\vspace{0.5cm}\\
\textbf{Descripción de la aplicación:} La lógica difusa se utiliza para controlar el movimiento de robots en entornos complejos. Los robots suelen estar equipados con sensores que les permiten recopilar información sobre su entorno. Esta información se utiliza para generar reglas difusas que describen cómo el robot debe moverse para alcanzar su objetivo.
\vspace{0.5cm}\\
\textbf{Implementación de la lógica difusa:} La lógica difusa se implementa utilizando un conjunto de reglas difusas, un conjunto de funciones de pertenencia y un método de inferencia. Las reglas difusas describen la relación entre las entradas y las salidas del sistema. Las funciones de pertenencia representan la forma en que los valores de las entradas y las salidas se distribuyen en el espacio difuso. El método de inferencia se utiliza para combinar las reglas difusas para generar una salida.
\vspace{0.5cm}\\
\textbf{Ejemplo:} Un robot se utiliza para limpiar un piso. El robot está equipado con sensores que le permiten detectar la suciedad en el piso. La lógica difusa se utiliza para generar reglas que describen cómo el robot debe moverse para limpiar el piso de manera eficiente. Por ejemplo, una regla difusa podría ser:
\end{document}
