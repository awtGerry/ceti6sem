\documentclass{article}

\usepackage[utf8]{inputenc}
\usepackage[T1]{fontenc}
\usepackage[english]{babel}
\usepackage[document]{ragged2e}

\usepackage{amsfonts}
\usepackage{natbib}
\usepackage[dvipsnames]{xcolor}
\usepackage{graphicx}

\usepackage[fleqn]{amsmath}
\usepackage{amssymb}

\title{Glosario de clases}
\author{Victor Gerardo Rodríguez Barragán}
\date{05 de septiembre de 2023}

\begin{document}
\maketitle
\begin{center}
    \includegraphics[width=0.8\textwidth]{/home/gerry/Documents/ceti/logo.png}
\end{center}
\justify
\newpage
\tableofcontents
\newpage

\section{Practica 01 - Notificaciones}
\subsection{MainActivity.java}
Es la activity que se crea por defecto en un nuevo projecto en Android studio
en esta activity yo realizo las conexiones hacia otras activity y hago la autentificacion
del login.

\subsection{DashboardActivity.java}
Es la activity que se crea por defecto en un nuevo projecto en Android studio
en esta activity yo realizo las conexiones hacia otras activity con el uso
de botones, uno es para mandar una notificacion y el otro para realizar un
cierre de sesion.

\subsubsection{onClick}
Es un metodo que se utiliza para realizar una accion cuando se presiona un boton
en este caso solo utilizo el boton de logout para cerrar sesion, para mandar la
notifcacion utilizo un metodo que se llama \textbf{displayNotification} el cual
crea una notificacion llamando a la clase \textbf{NotificationHelper} y le pasa
como parametros el titulo de la notificacion y el mensaje que se va a mostrar.

\subsection{NotificationHelper.java}
Es una clase que se crea para poder crear notificaciones, en esta clase se
defino los id de las notificaciones y canales, tambien se definen los metodos
por defecto para crear una notificacion.

\subsubsection{IDs}
Para la seccion de IDs tenemos que tener 2 la ID para notificaciones con:
\textbf{NOTIFICATION\_CHANNEL}: Es una variable constante que sirve para identificar el canal o el ``medio'' por el cual se agrupan y configuran las notificaciones.
\textbf{CHANNEL\_ID}: es una variable constante que se utiliza para identificar
el canal o el ``medio'' por el cual se agrupan y configuran las notificaciones.

\subsubsection{createChannel()}
Es un metodo que se utiliza para crear un canal de notificacion, en este metodo
se definen las caracteristicas del canal como el nombre, la descripcion y la
importancia de las notificaciones que se van a mostrar en este canal, asi como
otros parametros para personalizar el canal, aunque primero que nada se verifica
que la version de android sea mayor o igual a la version Oreo.

\subsubsection{getChannelNotifcation(title, message)}
Es un metodo para agrupar las notificaciones en un canal, en este metodo se
definen las caracteristicas de la notificacion como el titulo, el mensaje, el
icono, el sonido, la vibracion, el color, la prioridad, la visibilidad, etc, creo
uno de estos por cada canal que se va a utilizar.

\subsubsection{setPendingIntent()}
Es una funcion que se utiliza para realizar una accion cuando se presiona una
tecla dentro de la notificacion, en mi caso solo existen las opciones de
\textbf{SI} si es algo urgente y \textbf{NO} si no es algo urgente, en caso de
presionar \textbf{SI} se abre la activity de \textbf{EmergenciesActivity} y en
caso de presionar \textbf{NO} se abre la activity de \textbf{RegisterActivity}.

\subsection{EmergenciesActivity.java}
Es una activity simple, pensada en que se va a utilizar para registrar a
un paciente en caso de que sea una emergencia, por lo tanto contiene
un formulario corto y vital para poder dar atencion urgente al paciente.

\subsection{RegisterActivity.java}
Es un formulario para registrar a un paciente, en este formulario se piden
datos como el nombre, la edad, el sexo, el tipo de sangre, entre otras cosas
que son importantes para poder dar la atencion adecuada al paciente, en este
caso mando una notificacion extra para avisar que se ha registrado a un paciente
y pronto se le dara atencion.

\section{Practica 02 - Alarmas}
En esta practica se utilizo el mismo codigo que en la practica 01, solo se
realizo una modificacion en la clase \textbf{RegisterActivity}, en esta clase
se agregaron dos botones para poder programar una alarma, uno para programar
uno establece la fecha y el otro para establecer la hora.

\subsection{setDateAlarm()}
Metodo que se utiliza para establecer la fecha de la alarma, en este metodo
se usa la libreria \textbf{Calendar} para poder seleccionar la fecha y se
muestre al usuario en un \textbf{DatePickerDialog} el cual es un calendario
simple.

\subsection{setHourAlarm()}
Muy similar al metodo anterior, solo que en este caso se utiliza la libreria
\textbf{TimePickerDialog} para poder seleccionar la hora de la alarma.

\subsection{programAlarm()}
Es un metodo que se utiliza para programar la alarma, se pasan como parametros
extra el mensaje, la hora y los minutos en el que se realizara la alarma,
acto seguido se abre como ``activity'' el programa de reloj de android con
la alarma ya programada.

\section{Practica 03 - Firebase}
En esta practica se utilizo el mismo codigo que en la practica 01 y 02, solo se
realizo una modificacion en la clase \textbf{RegisterActivity}, en los cuales
se agrego un boton para poder guardar los datos del paciente en la base de datos
de \textbf{Firebase}.

\subsection{token}
Es una variable que se utiliza para guardar el token del dispositivo, este token
se obtiene de la consola de \textbf{Firebase} y se guarda en la variable para
poder mandar la notificacion a ese dispositivo.

\subsection{sendPushNotification()}
Es un metodo que se utiliza para mandar una notificacion a un dispositivo usando
\textbf{Firebase}, en este metodo se definen las caracteristicas de la notificacion.

\subsection{getHeaders()}
Es un metodo que se utiliza para obtener los headers de la notificacion, en este
caso solo se utiliza el \textbf{Content-Type} y el \textbf{Authorization}.

\subsection{onNewToken()}
Es un metodo que se utiliza para obtener el token del dispositivo, en este metodo
se guarda el token en la variable \textbf{token}.

\subsection{saveToken()}
Es un metodo que se utiliza para guardar el token en la base de datos de
\textbf{Firebase}, en este metodo se obtiene el token y se guarda en la base
de datos.

\subsection{onMessageReceived()}
Es un metodo que se utiliza para recibir una notificacion, en este metodo se
obtiene el titulo y el mensaje de la notificacion y se muestra en un \textbf{Toast}.

\end{document}
